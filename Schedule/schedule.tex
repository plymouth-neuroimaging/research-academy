\documentclass{article}
\usepackage{tikz}
\usepackage{datetime}

\title{Academy Research Day}
\newdate{date}{28}{07}{2023}
\date{\displaydate{date}}

\begin{document}
\maketitle


% These set the width of a day and the height of an hour.
\newcommand*\daywidth{6cm}
\newcommand*\hourheight{3.8em}

\centering

\tikzset{entry/.style 2 args={
    xshift=(0.5334em+0.8pt)/2,
    draw,
    line width=0.8pt,
    font=\sffamily,
    rectangle,
    rounded corners,
    fill=blue!20,
    anchor=north west,
    inner sep=0.3333em,
    text width={\daywidth/#2-1.2em-1.6pt},
    minimum height=#1*\hourheight,
    align=center
}}

% Start the picture and set the x coordinate to correspond to days and the y
% coordinate to correspond to hours (y should point downwards).
\begin{tikzpicture}[y=-\hourheight,x=\daywidth]
    % First print a list of times.
    \foreach \time/\ustime in {14/14:00,15/15:00,16/16:00,17/17:00}
        \node[anchor=north east] at (0,\time) {\ustime};

    % Write the entries. Note that the x coordinate is 1 (for Monday) plus an
    % appropriate amount of shifting. The y coordinate is simply the starting
    % time.
    \node[entry={0.85}{1}] at (0,14) {
        \textbf{DICOM introduction}\\
        \textit{\footnotesize{Lead: Mark Thurston}}
    };
    \node[entry={0.85}{1}] at (0,15) {
        \textbf{Python introduction}\\
        \textit{\footnotesize{Lead: Hongrui Wang}}
    };
    \node[entry={0.85}{1}] at (0,16) {
        \textbf{Neural networks introduction}
        \\\textit{\footnotesize{Lead: Megan Courtman}}
    };

\end{tikzpicture}
\end{document}
