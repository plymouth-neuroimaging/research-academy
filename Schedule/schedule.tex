\documentclass{article}
\usepackage{tikz}

\begin{document}

% These set the width of a day and the height of an hour.
\newcommand*\daywidth{6cm}
\newcommand*\hourheight{3.2em}

\tikzset{entry/.style 2 args={
    xshift=(0.5334em+0.8pt)/2,
    draw,
    line width=0.8pt,
    font=\sffamily,
    rectangle,
    rounded corners,
    fill=blue!20,
    anchor=north west,
    inner sep=0.3333em,
    text width={\daywidth/#2-1.2em-1.6pt},
    minimum height=#1*\hourheight,
    align=center
}}

% Start the picture and set the x coordinate to correspond to days and the y
% coordinate to correspond to hours (y should point downwards).
\begin{tikzpicture}[y=-\hourheight,x=\daywidth]
    % First print a list of times.
    \foreach \time/\ustime in {14/14:00,15/15:00,16/16:00,17/17:00}
        \node[anchor=north east] at (0.95,\time) {\ustime};

    % Draw some day dividers.
    % \draw (1,13.5) -- (1,18);
    % \draw (2,13.5) -- (2,18);

    % Start Monday.
    \node[anchor=north] at (1.5,13.5) {Fri. 28th July};
    % Write the entries. Note that the x coordinate is 1 (for Monday) plus an
    % appropriate amount of shifting. The y coordinate is simply the starting
    % time.
    \node[entry={0.85}{1}] at (1,14) {DICOM};
    \node[entry={0.85}{1}] at (1,15) {Python};
    \node[entry={0.85}{1}] at (1,16) {Neural networks};

\end{tikzpicture}
\end{document}
