% SPDX-FileCopyrightText: 2023 Mark Thurston
%
% SPDX-License-Identifier: Apache-2.0

\documentclass{beamer}

% https://deic.uab.cat/~iblanes/beamer_gallery/index_by_theme.html
\usetheme{Copenhagen}
% https://deic.uab.cat/~iblanes/beamer_gallery/index_by_color.html
\usecolortheme{beaver}

\setbeamertemplate{items}[circle]

% font config (https://tug.org/FontCatalogue/)
\usepackage[T1]{fontenc}
\usepackage[usefilenames,DefaultFeatures={Ligatures=Common}]{plex-otf} %
\renewcommand*\familydefault{\ttdefault} %% Only if the base font of the document is to be monospaced
% for GitHub icon
\usepackage{fontawesome}
\usepackage{tikz}

% UK date format
% https://anorien.csc.warwick.ac.uk/mirrors/CTAN/macros/latex/contrib/datetime2
\usepackage[useregional]{datetime2}

\usepackage[backend=biber, style=authoryear]{biblatex}
\addbibresource{bibliography.bib}

% title page details %%%%%%%%%%%%%%%%%%%
\title{Introduction to DICOM}
\subtitle{Academy Research Day}
\author{Dr Mark Thurston}
\institute{\href{https://github.com/plymouth-neuroimaging/research-academy
    }{\faGithub{} plymouth-neuroimaging}
}
\date{\DTMdisplaydate{2023}{07}{28}{Friday}}
\logo{\includegraphics[width=2.5cm]{images/uoplogo.jpg}}
\titlegraphic{
    \begin{tikzpicture}[overlay,remember picture]
        \node[below=1cm] at (current page.north){
            \includegraphics[width=2cm]{images/dicomlogo.jpg}
            };
    \end{tikzpicture}
}
%%%%%%%%%%%%%%%%%%%%%%%%%%%%%%%%%%%%%%%%

\begin{document}

\begin{frame}
    \titlepage{}
\end{frame}

\begin{frame}{Outline}
    \begin{itemize}
        \item{Clinical relevance}
        \item{File format}
        \item{Network protocol}
        \item{Examples}
        \item{Hands-on practice}
    \end{itemize}
\end{frame}

\begin{frame}{Clinical relevance}
\end{frame}

\begin{frame}{File format}
    \tiny\textit{{``A key feature in the design of the PACS is that the image
    data sent from the modalities and stored in the archive are in a format
    that can be recognized and used within the complete system. The image file
    contains the basic digital data that allow it to be displayed, but it also
    contains essential information including the imaging modality, annotations,
    display preferences, the patient’s name and other identifying data. The
    presentation of these data must be in a standard format, and the standard
    used is DICOM. The standard is complex. \ldots{}}''
    }\cite[88]{allisy-roberts_farrs_2008}
\end{frame}

\begin{frame}{Network protocol}
\end{frame}

\begin{frame}{Examples}
\end{frame}

\begin{frame}{Hands-on practice}
\end{frame}

\begin{frame}{References}
    \printbibliography{}
\end{frame}


\end{document}
